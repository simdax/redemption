Anoy s'enfonça lentement dans le temple. Avant de devenir nus,
les murs étaient couverts d'énormes bibliothèques. Les étagères
s'adaptaient aux tailles des livres

Anoy s'étonna du nombre de flèches que son carquois avait pu
déversé. Les pieds trempant maintenant dans l'eau jusqu'à la cheville,
les flèches qui s'étaient déversés dérivaient toutes dans la direction
de l'obscurité, happées dans des constellations traçant différentes
formes. Parfois les flèches semblaient indiquer un danger, une forme
menaçante, une chimère bondissante, un ours sauvage. Parfois les
flèches traçaient des cartes entières, Anoy reconnaissant son pays
natal, et d'autres lieux qu'il avait pu traverser ces trois derniers
mois. Les flèches se mouvant tout autour de lui glissaient parfois
doucement vers la droite, parfois se séparaient en plusieurs groupes.

Plus il avançait, plus le niveau d'eau montait et Anoy se retrouva
bientôt à devoir nager. Enfin, cet effort fut rendu vain, parce qu'il
vit que le niveau montait de plus en plus vite, et qu'il était amené
vers le plafond invisible du temple.

Anoy pensa que son heure était venue. Fasciné par l'idée de mourir
noyé dans un temple prestigieux, il se mit sur le dos, et se laissa
bercer par les flots, qui, sans aucun mouvement, montaient
progressivement.

La mort par noyade n'était pas sans noblesse. Il imagina son corps
inerte se figer dans le reflux perpétuel des ondes, au terme d'une
bataille qu'il espèrait courte. Comment allait-il se comporter au
moment fatidique? 

La plupart des flèches coulaient plutôt que de rester à la
surface. Elles disparaissaient plutôt que de rester à flot, ce que ne
remarqua pas Anoy avant de remarquer que la lumière ne provenait plus
d'elles, mais de la faible lumière des vitraux qui se rapprochaient
lentement de lui.

Pensant qu'il n'y avait rien à faire si soudain son corps frappait le
plafond, il n'y avait aucun intérêt à se débattre. L'espoir mal placé
est absurde, immoral s'il ne reposait que sur une petite minute qui
serait le temps maximal qu'il pourrait passer sous l'eau pour trouver
une issue de secours, dont rien ne semblait indiquer la présence.

Si le plafond était placé à, disons, cent ou deux cent mètres, combien
de temps lui restait-il à vivre ? Ne faudrait-il pas qu'il consacre ce
temps restant à trouver un moyen pour s'en sortir ? Il faudrait déjà
savoir à quelle vitesse le niveau monte actuellement.

Anoy fut tiré de ses réflexions par une nouvelle sensation de
piquotement. Croyant que c'était une de ses flèches qui l'avait piqué,
comme c'est parfois le cas lorsque des petites fentes déchirent un
carquois de mauvais qualité, il se retourna pour voir à quelle
distance se trouvait le fond, maintenant. Ne voyant rien d'autre qu'un
ciel maritime constitué de flèches luisantes, il pensa que peut-être
c'était Péluvin qui s'était réveillé. Il vérifia s'il était
toujours dans sa poche, et vit que ce dernier était devenu unu éponge,
ronde et glonflée. Les yeux bouffis s'ouvrirent, et comme un poisson,
il s'échappa des mains d'Anoy.

Trop surpris pour le poursuivre, et voyant qu'il s'enfonçait de toute
façon trop profondément, Anoy se remit sur le dos et reprit le cours
de ses pensées. Les fenêtres s'approchaient maintenant, elles qui
devaient être à une cinquantaine de mètres, ou peut-être moins ? Leur
grande tâche lumineuse s'agrandissait lentement, c'était la seule
certitude.

En fermant les yeux, Anoy se laissa charmer par le chant rond de
l'eau, inarticulé, d'une seule grande mesure aussi longue que
grande. Il sortit lentement de son enveloppe corporelle, tenta de
retrouver les sensations qu'il avait enfant dans son bain. Il mit la
main dans ses cheveux pour les sentir flotter et ferma les yeux. Au
lieu de voir le noir, ses yeux était remplis de tâches qui
nageaient. 

Les fenêtres s'approchaient maintenant de très près. Il s'approcha
d'elle pour en sentir la matière. C'était une sorte de matière boisée
translucide, qui, cela le rassura, était résistante. Par dépit de
conscience, Anoy mit un grand coup de coude dedans, ce qui n'eut pas
pour effet de la briser. Au moment du choc, les différentes cordes de
la lyre qu'il portait autour de lui se mirent à vibrer d'elle-même. Il
sentit à nouveau un mouvement dans son dos, et en se retournant, vit
qu'une foule de petits poissons, que la lumière de la fenêtre
éclairait, le suivaient. L'eau s'était peuplée de tout un monde
maritime, autour des flèches qui semblaient concentrer l'affluence
poissonière.

Les petits poissons, venaient s'y frotter, et certains d'entre eux y
laissaient quelques écailles en se coupant parfois sur tout leur long.

Anoy frappa une seconde fois contre la vitre, et la musique déclenchée
par la lyre, comme la première fois, se transforma en voix 

``Se laisser vivre, n'est-il pas se laisser mourir? Et se laisser
mourir ne reviendrait-il pas à se suicider ?''

