La petite porte donnait sur un court corridor au bout duquel le halo
doré entraperçu ondulait encore. Les flèches qu'il avait gardés dans
son carquois frottaient en chuintant contre 
les murs. Derrière lui, la
porte que la créature n'avait pas refermée fit résonner quelques cris
d'oiseaux nocturnes. Cet espace confiné et sombre rappela à Anoy
quelques nuits à la belle étoile qu'il avait passé terré dans des
champs qu'il désertait au petit matin, de peur que les paysans du coin
ne le trouvassent endormi au milieu des maïs ou des blés. Puis, Anoy
sent une puissante odeur de moisissure se dégager du premier virage
qu'il prend. Cette mixture de viande, de chair et d'épice, est forte
et puissante, mais pas désagréable, comme une odeur qui rôtirait de
plus en plus sans jamais brûler.  Les paysages qui ont accompagné ses
longues marches de ces trois derniers mois se rappelent et
s'effacent. Le souvenir du ciel immense au-dessus de lui dont il
tentait de fuir la froideur s'évapore ; les nuits à la belle étoile au
silence menaçant, la douceur feuillue d'une ligne de crête remuant au
vent, et l'harmonie du va-et-vient d'une forêt lui sont maintenant
rendues transparentes par la lumière hypnotique qui se dégage au bout
du tunnel. Dans une Pressant son pas contraint dans ce petit corridor,
ses rêveries sont soudainement recouvertes par une giclée violente.

Anoy éternue en rejetant une poudre multicolore. Il vient de recevoir
sur la tête un mélange de matière molle et odorante ; une échoppe,
dont les cuisines recrachent par paquet entiers ses victuailles
invendus, trop cuites ou fermentées dans les égoûts dont sort

--- Chaud devant, gardez moi les meilleurs morceaux, et jetez les
entrailles avec la soupe !

Alors qu'il se précipite hors de son trou avant de se faire
ébouillanter, il tombe nez à nez avec un petit singe couvert d'un
chapeau gigantesque et pointu, qui ne s'étonne nullement de la
présence inopiné d'un grand brun encapé, portant carquois en recouvert
de nourriture. Il déversa prestement sa bouilloire, et retourna
assister une cohorte d'autres singes découpant ciboule, farcissant
vollaile et sucrant fraises par dizaine dans les moindre recoins. Anoy
était arrivé dans une cuisine en pleine activité, dont on voyait par
une porte qu'elle servait à nourrir une assemblée de citoyens
affamés. De l'ouverture de la porte se dégagea un mélange
incompréhensible de sons et d'odeurs qui couvrit celles déjà présentes
dans la cuisine, ainsi qu'un cuisinier, qui, avant de sauter pour
réceptionner en vol une casserole dont le contenu se mettait à buller,
voyant Anoy, finit par dire:

--- Tu es là pour te reposer ?

--- Non, non, pas du tout. Malgré l'incongruité de la questions, Anoy
avait répondu ce qui lui semblait le plus approprié et le moins
rempli d'émotions.

--- Qui es-tu, alors ? On est là pour se reposer, alors, si tu as envie
de plomber l'ambiance en te concentrant là, planté comme une statue,
passe ton chemin !

Et le cuisinier barbu de se jeter sur un évier rempli à ras bord de
fonte et de graisse pour se mettre à gratter énergiquement et faire le
plus de mousse possible, en envoyant Anoy directement à la sortie d'un
coup de pied énergique dans les fesses.

La salle à manger
Peinte à la gouache, la ville florissante crache son pollen au visage
d'Anoy. Un petit vertige saisit ce dernier qui ne parvient plus à
distinguer aucun tracé. Les couleurs des bâtiments se chevauchent les
unes aux autres, . Les noms d'échoppes, les écussons et les emblèmes,
toutes ces marques qui délimitent un espace public et un espace privé,
ont disparu dans une explosion de forme sans fond, sans
silhouette. Tout indique que l'on est dedans. A l'intérieur.

C'est jour de fête ?  On lui répond que c'est simplement jour de
congé,

