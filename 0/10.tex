La cité de Du. avait pour réputation d'être un havre de paix.

Son absence de politique lui conférait un statut mal
défini. Contrairement aux multitudes de cité-états qui
l'environnaient, Du. n'avait pas d'indépendance politique, en même
temps qu'aucune nécessité à en avoir. N'ayant pas de ressources autres
qu'agricoles, n'étant pas sur une position privilégié et ne pouvant
créer aucun désir, la cité prospérait. Cette prospérité elle-même, ne
paraissant n'avoir aucun intérêt à se montrer, passait inaperçue. Son
manque d'ambition n'avait jeté à sa tête aucun représentant, et elle
était gouverné uniquement par la simple obéissance commune à la
prescription d'un Livre inachevé. 

Ce miracle premier n'était pas tant le lieu, certes admirable où se
trouvait le Livre, mais la dévotion naturelle que les premiers colons
prêtèrent à ses dires, tradition perpétuée sans aucune pression, ni
politique, ni pédagogique, ni théologique. La beauté de ces premiers
croyants étaient leur profonde humilité à voir dans le spectacle d'une
nature particulière une véritable écriture, et dans la croyance
profonde que ces écrits n'étaient pas interprétés, mais provenaient
d'une langue existant en soi, extérieure à eux, et que la curiosité
intellectuelle ne pouvait s'empêcher de décrypter. 

Ce dernier était réactualisé selon certains signes de la nature, une
moisson précoce, le brâme d'un animal lointain et mythique, une lune
plus rousse en constituaient une infime partie. La culture, le
quotidien, le bien-être, tout était dirigé par une pensée commune qui
ne survivait jamais à la sortie en-dehors de la ville, comme si le
monde extérieur était entier condamné à vivre selon un état de Nature
incapable d'en absorber le moindre enseignement. En réalité, la simple
présence de la cité sur la carte d'autres peuples était en soi une
bizarrerie, tant elle avait peu à partager avec elle.

Pourtant, il était une spécialité qui avait fini par répandre la
réputation de la cité hors de son territoire. Il y était connu qu'on
pouvait y guérir l'envie de mourir. Dans cette époque lointaine, cette
maladie ne touchait pas peu de monde. Les différentes instabilités
politiques, les fréquentes mauvaises récoltes, le sentiment diffus que
la place venait à manquer, en même temps qu'un climat de violence
global permettait à chacun de se sentir autorisé à commettre une sorte
d'acte de légitime défense. Un corps acculé dans ses derniers
retranchements, ayant perdu toute dignité et pourtant mû encore par
une volonté qui s'efface de plus en plus à la conscience, celle de
vivre ou de tuer. La première option fuyant, elle choisit
naturellement la seconde.

Ce n'est pas sans tristesse qu'Anoy avait décidé de fuir le cercle
parental, mais c'était la seule solution qu'il avait trouvé à son
mal-être. La famille avait une morale et une hygiène vitale stricte,
mélangé à une libéralité totale dans les moeurs. Pensant peut-être
avec naïveté qu'une bon départ dans la vie ne pouvait mener qu'à un
tracé sans détours, il y avait dû y avoir des manques dans l'éducation
tardive d'Anoy, le moment dangeureux, où devant se donner ses propres
maximes mais n'ayant pas encore l'expérience, et plus vraiment la
capacité enfantine de s'adapter dans un sens vitalement neuf à tout
déboire, la protection s'était relâchée là où elle aurait dû tenir bon
encore un temps.

Une mélancolie infinie m'a pris à ce moment, avoua-t-il au conducteur
barbu, sans explication. Il s'en suivit mécaniquement une
intériorisation du monde et une incapacité grandissante à trouver de
l'intérêt. Les choses se révélaient sans esprit, et l'intelligence
s'ennuyait. En même temps, les quelques pointes brillantes dans le ciel
de l'esprit devenaient brûlantes, inatteignables, presque
mauvaises. 

C'était donc un mauvais moment pour s'investir entier dans une
passion. 

Mais qu'est-ce que tu pouvais y faire ? Il y a une physique aux
phénomènes d'ennui et d'attirance. Moins il y a de bougies qui
brûlent, plus chacune brille fort. Tu pouvais donc bien essayer de les
éteindre, le niveau de lumière ne baissait jamais. à moins,
croyais-tu, de pouvoir toutes les éteindre.

Le barbu rit aux éclats

Ah ah, tu t'y es super mal pris ! Dans l'endroit où on t'emmène, il
n'y a pas ce genre de passion.

Comment est-ce possible ?

Si on continue de filer la mèche, disons 


Les méchants sont ceux qui regrettent d'avoir vécu au seuil de la mort.

