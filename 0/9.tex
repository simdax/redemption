Cette dernière pensée lui fit 



Du fond des eaux surgit un serpent de mer dont mille convolutions
écrivaient dans l'eau un milliers de signes, rappellant chacun une
personne connue, un chose vue récemment, une conversation ou une
simple interjection. Tout l'espace du bassin se remplit de chair
serpentine, les gouttes deviennent des écailles, et sur la tête du
serpent géant, trône en silence une courbe humaine, féminine, sans
forme ou rondeur. La déesse sans visage porte un masque descendant
jusqu'au nombril, formant un menton pointu et blanc ivoire. Le serpent
de mer se révèle avoir la tête du chien de l'entrée du temple, au
regard perpendiculaire.

Les deux visages se tiennent immobiles dans l'entremêlement des
écailles et les noeuds du corps du serpent. Aucune souplesse ne se
dégage de ce mouvement. Il semble plutôt provoqué par
l'entrechoquement logique de chaque écaille, une à une, à la vitesse
du son, dans une harmonie numérique perpétuelle.


Tu sais que tu n'as pas le droit de te suicider, et comment peux-tu
savoir si ce que tu commets est bien un suicide ? Si tu t'engages pour
une guerre meurtrière, une cause perdue, ou un voyage sans retour loin
de tes proches ? Toutes ces petites morts de la vie que l'on provoque
parfois sciemment, sans même y penser, comme sont les ruptures, les
désillusions, les fins en général, forment-elles à leur échelle autant
de petits suicides ?

De même, si tu cherches uniquement à apaiser en toi une douleur qui
devient trop vive, si ton but est au fond la recherche d'un
remplacement, trouver quelque chose qui irait combler une
absence, trouver un plein à un vide, si tu cherches seulement à
ressuciter quelque chose, ne serait-il pas paradoxal d'appeler cela
un suicide ?

La déesse refit claquer ses écailles en un millier d'éclats

Un des groupes de flèches qu'il avait suivi frappa contre un mur. Anoy
se retourna, et vit la porte d'entrée du temple minuscule, en-dessous
des quelques vitraux gigantesques qui la surplombait, au loin,
s'ouvrir en grand dans un grand fracas. Une grande lumière blanche
s'engouffra un peu plus dans la partie obscure qu'il parcourait, lui
révélant les tracés de ces grands murs vierges qu'il suivait à
l'aveugle. Des silhouettes lui semblèrent apparaître, engloblées dans
le halo de lumière, mais elles étaient trop petites pour qu'il puisse
les distinguer.

Pris de panique, Anoy se sentit happé par un courant qui le ramenait
vers l'entrée. L'eau baissait de 


Anoy fut retrouvé dans le coma ? Avait-il trouvé la mort qu'il
souhaitait si ardemment ?
