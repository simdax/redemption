Il crut sentir quelque chose remuer dans sa poche. Ses vêtements
gonflèrent par la bourrasque lorsqu'il fut gagné par un second souffle
qui le souleva et le fit trébucher. La porte du temple lâcha avec
grand fracas, le contraignant à y pénétrer malgré sa réticence. A. ne
put la refermer, et le souffle éteignit toutes les lampes. Guidé par
la seule lumière de la nuit, A. trouva une infractuosité dans
laquelle, malgré les efforts redoublés du vent qui faisait tomber
nombre d'objets saints, la tempête n'arriva pas à pénétrer. Ici, gagné
par le froid, il combattit l'effroi en tentant de se référer à
quelques divinités qui lui étaient inconnus, mais dont il aperçut le
regard des statues derrière lui.

Cette pluie absurde qui valse jusqu'à la nausée était la volonté de
son père, il en était persuadé. Les pouvoirs de ce dernier étaient
immenses, et il était peu de choses qui ne sembla incroyable à
l'ordinaire qu'il ne pût accomplir. A. tentait de le fuir, mais il
était là, toujours présent et pur, flottant dans l'air, nulle part et
potentiellement partout. A. désespérait de trouver un jour un moyen de
le semer. Se rendre malade, il y arriverait un jour, et déjà il
sentait qu'il ne se mouvait plus aussi gracilement qu'avant la
fuite. Si son père parvenait à le pousser au bord de l'épuisement,
s'il le réduisait au repos, A. n'aurait plus d'autre choix que de
revenir sous son joug. Il tentait bien de se cacher mais il finirait
bien un jour ou l'autre par être retrouvé. Il ne pourrait le fuir
indéfiniment.

Seule la mort semblait pouvoir échapper à cette puissance paternelle
despotique. Enlever à son père cette possession. Pourtant, au milieu
du chemin, A. se rattacha à une autre idée. Il réfléchit à trouver un
autre protecteur, qu'il penserait aussi puissant.

Seule la religion pouvait peser dans la balance de cette vie
fragile. La pensée du suicide aurait effleuré quelqu'un d'autre,
plutôt qu'une fuite incertaine dans les voies sacrées, et le
détachement de l'âme de l'enveloppe du corps était un projet plus doux
que cette difficile rédemption à laquelle se rattachait maintenant
A. L'idée de suivre une voie religieuse avait moins d'éclat que de
retrouver les paysages mythiques du vide éternel. Mais l'esprit de
A. était gardé par un Inconnu qui empêchait quelque idée que ce soit
de pénétrer la chambre de ses pensées. Même les délices infinies de la
mort s'y voyaient refuser l'entrée. La seule mort qui lui paraissait
digne était celle de l'ennui, celle qui fuyait au lieu de le
divertir. Elle devait le surprendre ; c'était peut-être la dernière
chose qui avait ce pouvoir, et il ne voulait pas gâcher cela en la
précipitant.

La religion était le garant de l'ennui à ses yeux. Mais il était une
raison cent fois plus valable encore à ce refus de mourir par
suicide. C'est qu'il est interdit aux gens de son espèce de le
commettre. A. faisait partie d'une famille âpre dont il n'était pas
clair qu'elle fût un regroupement de personnes ou une race à part
entière. Cette famille était régi par un code très strict qui
échappait aux conventions extérieurs. Cette engeance formaient depuis
toujours leur propre bon sens, au détriment de celui des bonnes
gens. Cette grande famille n'était pas grande, mais ne se
connaissaient pas, dispersés disproportionellement à leur nombre aux
quatre coins du monde.

Il leur était aussi interdit de montrer leurs émotions aux personnes
extérieures à leur famille. Comme des Titans, ils étaient en société
impassibles. Ils formaient habituellement les hommes de l'ombre des
différentes sphères politiques, car il était impossible de deviner
leurs pensées ; les politiques les craignaient quand ils apprenaient
qu'ils avaient à les affronter.

Le vent finit par se rendre moins glaçant. La fatigue gagna la
tempête. Aux espèces ludiques de drapées et aux attaques de couleurs
foudroyantes sur ses vêtements, le silence imposa sa paume sur les
paupières de la nuit. A. fut frigorifié lorsque le mouvement des
particules qui le berçait s'arrêta. Comme un voleur qui découvre que
la chambre qu'il pénétre est en fait vide, la tempête battait en lui
l'espoir dans lequel la contemplation de ce plein spectacle le
maintenait, mais, comme les gouttes s'assagissent et terminent leur
histoire, les trous de l'espace sautent à ces yeux. Les vestibules
peuvent se découvrir tandis que le raffut climatique s'estompe. Il
voit les dessins géométriques déjà aperçues sur les bâtiments
extérieurs, des représentations de Divinités qu'il lui étaient
inconnus. Lentement, en touchant le moins possible au silence qui
s'est soudain établi, il ressort vers la pierre mouillée de la ville.

A côté du temple, dans une infractuosité, alors qu'il s'essuyait le
visage avec sa cape en faisant détremper ses vêtements et qu'il se
défroqua discrètement, il sentit

////

Alors qu'il n'est plus contraint de rester ici, A. savoure le chant du
vent, puis, dans ce qu'il croit être une trêve, une voix intérieure
qui le plonge dans le temple.
