La lune dardait plus fort ses rayons qu'une épaisse couverture de
nuages filtre. Pourtant, la majorité de la lumière provient non plus
d'en-haut, mais d'ici-bas. Le toit céleste se rapproche et forme un
épais brouillard qui darde sous le rayon de ses soleils
terrestres. Tout entier dans cette nouvelle entrée au monde, Anoy
ouvre les yeux avec une grande souplesse, car l'heure avancée de la
nuit n'empêche pas la ville de s'agiter encore, au contraire, et pour
se mouvoir dans la foule qui s'agrandit au fur et à mesure que le
brouillard s'épaissit, Anoy est contraint de se contortionner,
d'éviter les uns, les autres, d'esquisser des pas chaloupés. Lui qui
avait perdu sa démarche natale, le voilà qui danse sous la
contrainte. Contrairement aux faubourgs, la méfiance des habitants du
centre est transformée en joie, informe et sans but apparent. Cette
bienveillance donne espoir à Anoy. Le plein total est par essence
unique, et tout vient s'entasser.

La musique qui emplit le paysage trace des chemins de danse
qu'empruntent une jeunesse dans laquelle Anoy ne reconnaît rien de ce
qu'il a vécu. Les gens crient, rigolent en rythme, s'exclament en
choeur, se répondent par contre chant et s'embrassent en accord
parfait. Pourtant, il comprend assez vite que cette joie ne laisse
que peu de place à l'étranger. 

Personne ne semble le voir, et même son enveloppe physique ne semble
un obstacle à personne. Il est bousculé, confondu, emporté dans des
rondes qu'il tente de quitter aussitôt, en vain. La langue s'est
dissoute en musique, et le sens une pure émanation d'harmonie. Les
rues

Un badaud lui tend
une choppe d'un breuvage acide et sucré, dont seuls les deux premières
gorgées atteignent son gosier avant de finir dans la main d'un singe
qui sautillent d'épaule en épaule. Même ses grognements, ses demandes répétés pour se frayer un
chemin, sont intégrés aussitôt dans un nouveau jeu dont tous se
réjouissent. Son accent est imité, devient aussitôt un nouveau mode
vocale, son accoutrement qu'autant les hommes que les femmes tentent
de lui arracher, une nouvelle mode vestimentaire.

Dans son étalement, son gaspillage
constant d'énergie, l'individu dissout ne peut s'inscrire qu'en
faisant partie de la culture locale. L'étranger en tant qu'étranger ne
peut qu'être condamné à s'y dissoudre, ou à ne pouvoir pénétrer.

Au loin, il devine les lignes hexagonales et les polyèdres des motifs
héraldiques. La beauté de ce nouveau paysage pieux ne s'inscrit même
pas dans sa tête.  Ces nourrices paraissaient chaudes vues de loin,
elles sont belles vues de près mais leur contact ne reçoit pas
l'étranger, qui souffre intérieurement de ce dédain.

Soudain, un doute le saisit à la vue des premières marques du sacré
que dispense au tout-venant depuis son coeur la ville. Il comprend et
redoute qu'en pénétrant à cet horaire indécent, il n'y eut pas de
place pour lui. Anoy s'arrête. Que lui reste-t-il à faire ?  A-t-il le
choix de reculer ? Derrière lui, quelques badauds semblent avoir
aperçu son physique différent. Quelle bonne question doit-il poser ?
Il avait fui les vices, mais il manquait encore de vertu.

Anoy se met à hésiter. La nuit n'est pas calme, le vent souffle
froidement. En ville, aucune végétation ne permet de dire ce que la
peau ressent. 

Seuls les regards de la roture dictent le temps. C'est
la cinglance des pensées qu'elle couvre d'une vue vénérable, tant
d'églises et de palais, autant de mères nourricières qu'il en faudrait
pour nourrir mille peuples. Quelques murmures arrivent à ses oreilles,
Anoy quitte les faubourgs et décide de fuir vers les places
centrales. Les enceintes impressionantes qui gardent des secrets
succèdent aux rues banales qui tentent de montrer ce qu'elles n'ont
pas.


Elle est trop lourde: il baisse les yeux dans ses larmes et il plie
les genoux, au milieu de la place. Il reste quelques minutes vide, en
détresse, au milieu de la pierre blanche que la nuit rend
grise. Personne ne le voit dans son désert. Et puis, il entend le
ciel gronder. Quelques fines gouttes s'attardent pour lui remonter le
moral. Il relève son chef et ses yeux humides mouillent les
nuages. Mais bientôt, le vent le chasse, et il n'a plus le choix que
de se battre contre l'avant qui le refuse et l'arrière qui le
repousse.

Sa cape trouée se plaque fortement contre son corps comme une femme
qu'il quitte. Il marche près des colonnes d'Aphrodite qui se dressent
à ses côtés, à l'abri de laquelle il arrive presque en rampant. Il
caresse une colonne qui s'élève de la nuit pourtant. La bandaison de
son arc frotte sa jambe lorsque sa main appuie sur la pâleur de la
pierre et elle marque sa cuisse avec vigueur. Il pousse un léger râle
qui ne s'oublie dans la musique de la ville.

Il attend un peu avant d'aller plus avant et il pose son dos contre
les courbes religieuses de l'édifice. Cet endroit qui ne dit pas
bienvenu le rend maussade. Le but de son voyage n'est plus clair. Il
tente de se rappeler les mots qu'avait prononcé l'étranger avant qu'il
ne reparte avec son cheval et sa charette, mais il a erré trop
longuement avant de parvenir jusqu'ici. Dans un temps perdu et lourd
qui l'a estompé dans le néant qu'emporte dans une bourrasque nouvelle,
un coup de vent terrible le gifle. Le ciel regronde. Des gouttes
recouvrent ses cils et s'engouffrent dans ses habits, terriblement
froides, dans les trous du tissu. Anoy tente de recouvrir ses
interstices, mais il perd à ce jeu de vitesse. Il se met à tousser et
s'étourdit en tentant d'éviter la pluie.

D'autres gouttes, plus lourdes et moins naïves, viennent s'abattre. Il
les sent sur la pointe de son nez, comme un rhume. Par on ne sait quel
tour, les gouttes finirent par trouver le moyen de contourner les
premières colonnes de l'entrée du temple, et contraignent Anoy à se
réfugier jusqu'au pied de la porte, solidement fermée et humide
sous le poids de son carquois qui se vide de l'eau qui l'avait
pénétré. Il n'avait pas senti son coeur battre depuis longtemps. Ce
souffle nouveau emplit tous les espaces de son corps. Cette joie
nouvelle lui est inconnue.

Il se souvint alors de la phrase du conducteur. "Les méchants sont
ceux qui regrettent d'avoir vécu lorsque la mort se présente."
