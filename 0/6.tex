Le petit enfant de la statuette s'excava de la poche, et secoua ses
petites ailes desquelles s'échappèrent dix étincelles. A. eut un choc
modéré à ce spectacle, tout étonnant qu'il fut. Mais, quand les dix
étincelles se posèrent sur chacun des doigts de A., en sentant la
chaleur vive et pénétrante de ses petites étincelles qui remontèrent
droit dans sa colonne vertébrale, il n'eut d'autre réflexe que
d'attraper la fée dans ses mains, comme saisi d'un désir
incontrôlable. Il eut la chair de poule, de chaleur et d'émotion en
attrapant jusqu'à la brûlure la fée, qui, terrifiée de se retrouver
dans des mains sales et trempées, finit par devenir insupportablement
électrique. A. détacha ses mains et laissa s'échapper l'être féérique
qui n'était autre que la petite statuette qu'il avait ramassé tout à
l'heure. Il revit le sexe énorme de la fée s'enfuir pendant et
laissant derrière lui une pluie d'étoiles délicieusement brûlante. Le
voyant appeuré et hors d'atteinte, A. agitait ses bras en vain pour
retenir la fée qui lui échappait. Ses étincelles furent de plus en
plus brûlantes. La fée qui tournoya en hésitant de prime abord prit de
la hauteur. Avec grand calme, les mots sortirent de la bouche de
A. sans qu'il ne remarqua qu'ils étaient les premiers mots qu'il
prononçait depuis des jours dans sa langue natale : "S'il vous plaît, restez".


Surpris par la grâce de la voix de cet hôte inconnu, l'enfant au sexe
démesuré suspendit son envol. D'une voix inarticulée de bébé, l'être
féérique prononça ces paroles : "Mévier inconnus, quand pouboir me
faire fortir rêbe de pierre."  "Je suis désolé, je n'ai pas fait
exprès de te sortir de ton sommeil, petit enfant."

A l'écoute de ces mots, l'être féérique se mit à dégager une lumière
rouge intense qui éclaira le temple jusqu'à son toit infini. "Ne fuis
pas un enfant !" se mit à rugir l'être dont la voix prit un timbre des
plus virils.

L'explosion de colère de l'être féérique dégagea une chaleur
incroyable. Chacune des chairs de A. se sentirent comme fondre dans un
frisson. De sa peau qui sécha en une seconde à ses os fatigués par de
longues errances, au bord de la rupture, qui s'étaient solidifiés
soudainement, ses vicères pâteuses étaient devenues plus légères, et
sa faim, sa soif, ses maux en tout genre, s'évaporèrent d'un seul
coup. Le choc de son crâne contre le sol ne laissa qu'un bruit sourd
se répandre dans l'obscurité qui regagnait avec parcimonie le temple.

Couché, l'être féérique bondit sur A. et se plaça à quelques
centimètres de son visage. Devant le spectacle du pénis énorme de la
fée, A. détourna le visage, mais cette dernière ramena son visage en
face d'elle et lui dit, une main sur son menton et une autre soutenant
son engin énorme autour de sa taille :

-- Toi quitter ton pays ?

La question surprit A. La fée, qui avait repris sa voix d'enfant,
reposa sa question:

-- Pour quoi vouloir quitter pays ?

-- Que... que signifies-tu par là ?

En posant cette question, A. remarqua que la poitrine de l'être
féérique commençait lentement à acquérir une opulence qui chatouillait
son nez. Au grossissement de sa poitrine avait correspondu un net
rétrécissement de l'appendice que le bébé volant, de plus en plus
féminin, ne tenait maintenant plus, mais qui restait à distance
respectable. Avec sa main libérée, la fée tira une mèche de A :

-- Pas faire l'idiot !

Chaque fois que la créature le touchait, toujours cette violence
remplissait A d'un bien-être incroyable qu'il voulait faire durer
encore malgré l'évident écoeurement qu'il ressentait au contact d'une
créature pansexuelle.

-- Qui êtes-vous ? Vous êtes une créature de mon père ? Que me voulez-vous ?

-- Non ! Fuis Péluvin, et agis moi-même. Fi tu beux aide, ftop
   queftions ftupides.

-- Pour quelles raisons exactement voudriez-vous m'aider ? Nous
   n'avons jamais été présentés à ce qu'il me semble.

Les seins de Péluvin grossissaient encore et ce qui était un
face-à-face entre eux finit par être recouvert de deux ballons
gigantesques que Péluvin tentait vainement d'écarter:

-- Dépêfe toi, pas temps ! Ftop réfléfir : Dire pourquoi
   quitter pays ?

La méfiance naturelle de A. l'empêchait de le dire.

-- C'est un monde de désolation que celui de l'incertitude. Rien
   n'y pousse, et on s'y assèche au dernier degré.

Pas décontenancée une seconde par l'aspect cryptique de la réponse
elfique de A. la féé rebondit :

-- Quel âge as-tu ?

-- Je l'ignore

-- Vrai? Beux faboir? Péluvin peut toi dire.

-- Non merci

A. tenta à nouveau d'attraper la fée, mais cette dernière se
retransforma en statuette brutalement aux derniers mots de A. Du
néant, une voix nettement plus féminine se répandit hors de la
statuette.

-- il n'est rien qui pourra te rendre la grâce de ta jeunesse ?

-- Je crains bien que non.

La fée durcit dans le même mouvement le coeur de A. qui se
recontracta, et oublia quasi instantanément la sensation que l'être
féérique lui avait procuré. La colère de A. s'en trouva redoublée, et
ses vêtements redevinrent humides sous le coup de son émotion.
