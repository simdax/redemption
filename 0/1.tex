Comment dépeindre en une seule image la foule de sentiments négatifs
qui peuplent une âme perdue au milieu d'une pègre qui l'ignore? Cet
énième taudis, le centième qu'il traversait en autant de jours, était
encore plus pauvre que les premiers qu'il avait traversé et qui avait
choqué sa naïveté d'alors. Dans celui-ci, on s'éclairait encore à
l'huile, bouffissant les rues d'une lueur végétale permettant avec
peine même à ceux qui avaient mémoire des lieux à se repérer et d'une
odeur qui se mêlait aux effluves de gras et d'alcool que consommaient
les quelques truands qui s'ignoraient encore debouts à cette heure de
la nuit. Ils guidaient Anoy à travers la matière de leurs ruelles dans
une langue brutale, à l'inimitié naturelle, contre menu monnaie ou
autres propositions moins décentes. Parfois, ils l'envoyaient
simplement paître dans quelques insultes, qu'Anoy, qui avait appris à
l'école le modèle linguistique de la région, comprenaient sans avoir
jamais appris. Souvent, il plissait ses oreilles pour tirer la
substance acoustique originale de ces borborigmes, mais au final, il
ne s'attardait jamais trop avec ce genre de personnes, et prenait un
chemin que le seul hasard avait décidé.

Comment décrire en un seul mot la foule de sentiments désagréables qui
peuplent une âme perdue au milieu d'une pays qui l'ignore?  Malgré
l'état pitoyable de ses vêtements, son port et sa démarche altière qui
s'était transformé par la fatigue en un pas de bourré et son accent
étrange, quelque chose faisait qu'Anoy n'était pas couleur
locale. C'était surtout son grand arc, qui attirait le regard amusé
des enfants qui se réveillaient par magie à la vision des abondantes
flèches qu'il transportait dans son dos, qui le distinguaient. Des
vieillards qui devaient maintenant recoucher les bambins, il tirait
les informations les plus certaines. Les femmes étaient aussi
grossières que les hommes, mais quelques unes tentaient parfois, pour
l'aider, d'articuler quelques mots dans ce qu'elle devinait être sa
langue, et il fut étonné de voir qu'il pouvait déclencher même dans
ces endroits sans éducation une discussion qui se voulait érudite sur
telle conjugaison ou tel usage. Par ailleurs, ces conseils étaient
rarement utiles, si ce n'est à éveiller lentement un sentiment de
jalousie dans tout le quartier, proportionnel au désir que peut
susciter un étranger jeune et sans défense.

Après une dizaine de zig-zags, il se vit offrir un lit par une femme
qu'il avait croisé une heure auparavant à l'endroit duquel il avait
fini par retourner. Quelques minutes plus tard, il se retrouvait au
milieu de nulle part après avoir couru pour semer un mari jaloux qui
n'avait aucun sens du partage, et il trébucha sur une petite statuette
assez grosse, de la taille de la main, qui gisait dans la
boue. C'était un petit enfant ailé au sexe disproportionné et
hilare. Il tendait sans pudeur ses fesses aux yeux d'Anoy. Rien ne le
rendait moins enfantin. D'abord un peu écoeuré par les atouts d'un tel
fétiche, Anoy finit par remarquer la qualité de la finition de l'objet
dont l'artisan local qui avait dû le produire ne pouvait
vraisemblablement que s'enorgueillir. La peinture notamment, rendait
sa biologie provocante presque réaliste, et le travail de texture
rendait à merveille les émotions de la statuette qui arrivait presque
à transmettre sa joie. Sur le socle d'icelle, était gravé une phrase
salace, qu'Anoy n'était capable de déchiffrer qu'après son séjour
intense dans l'arrière-boutique de la langue du pays : ``la raie dans
le fion''.

Elle semblait se moquer de quelque chose, afficher un certain mépris
pour ceux qui le trouvaient, peut-être. Sa main gauche tenait son
ventre, mais son bras droit tendait le doigt dans une direction. Au
bout d'un certain temps, Anoy remarqua qu'il pointait toujours le même
endroit, changeant même de bras si on s'amusait à le tourner. Etonné
sans se formaliser, Anoy s'amusa de ce signe du destin. L'automate ne
pouvait pas être un moins bon guide que ses derniers interlocuteurs;
tout en le serrant, il marche à bon pas sur le chemin de la
rédemption.

Les nombreuses nuits à dormir mal, dans un endroit inconfortable,
entre une nature inhospitalière et une société dangeureuse, les
bas-fonds, les taudis et ses habitants au visage buriné ou les
campagnes avec son lot d'odeurs putrides, qui lui ont donné un
torticolis et une nausée qui ne disparaissent pas depuis deux mois,
qui ont formé un quotidien repoussant les limites toujours plus au
bord de la maladie, il l'espère achevé maintenant qu'il arrive à son
but. Les images fatigantes des nuits passées ne valent pas d'être
remémorées, mais il est douloureux de se mutiler volontairement d'une
partie de sa vie. N'y a-t-il rien à garder de ces errances dans ses
bas-fonds, dans ces mondes parallèles qu'ils frôlaient sans le savoir
durant son enfance ? Ne peut-il pas trouver une issue au labyrinthe de
valeurs ésotériques que ces gens lui ont montré en étant eux-mêmes ?

Anoy avait vécu dans un isolement terrible depuis trois mois
maintenant. L'accumulation de rencontres de ces derniers temps avaient
fini de le dégoûter du commerce social. Doit-il regretter cette vie
privée de sens et de saveur ?  C'est la question qui liquéfie son
crâne et le fait bouillir d'une colère interdite. L'esprit en purée,
il se caresse avec rage le front pour soutenir cette pensée
intolérable. Les quelques passants qu'il rencontre n'ont plus l'air de
le remarquer, ni de se soucier de lui. La nuit avançant elle aussi
d'un pas certain, les passants s'égrainaient de plus en plus le long
des rues. Plus il emprunte de rues nouvelles et plus l'odeur
s'estompte, plus les gens l'évitent. Cette bouillie, son cerveau,
trouve de plus en plus de place pour s'exprimer et en certains
endroits, elle le pique.  Elle est la pointe, si aiguisée qu'elle en
est invisible, d'une épée qui le transperce dans tous les endroits
secrets du corps, et que tout le monde feint de ne pas voir.






