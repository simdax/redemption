Un rayon pénétra lentement dans le temple. Il hésita avant de se
refléter sur un calice renversé. D'une de ses fendures, de laquelle
suintait la nuit écoulée, le rayon ruissela ensuite jusqu'à la dorure
d'un cadre. Le tableau représentait une femme se baignant. Le rayon
plongea dans la peinture gardée par la statue d'un chien immense au
regard surplombant et perpendiculaire, la mer peinte immobile se mit à
onduler dans le reflet bien réel de l'humidité du temple, dont le sol
était entier recouvert d'une pellicule humide du souvenir de la
nuit. L'oeil sans émotion du chien s'éclaircit d'un deuxième rayon de
lumière finement parallèle, venu rejoindre son compagnon, qui vint
révéler un peu plus l'étendue du désordre.

Même si la tempête avait cessé, il semblait que le niveau
des eaux avaient légèrement monté depuis quelques minutes. La
statuette de Péluvin trempait maintenant jusqu'aux fesses, et ses
dernières gonflaient à mesure de son imbibation.

Anoy fut saisi devant l'étendue maritime, son esprit
divagua. Ayno saisit la statuette, et tenta vainement de la réanimer
en frottant avec répugnance les parties gonflées du bois. Il murmura
un "M'entendez-vous ?" qui se dirigea sans revenir jusqu'au fond du
temple. De petites vaguelettes de voix se répandirent en fendant le
filet d'eau à ses pieds de tracés ronds. Et le silence revint,
entrecoupé de nouveaux rayons d'or.

Anoy se leva et scruta l'étendu de l'espace sacré qui était plus grand
que l'obscurité nocturne ne l'avait laissé entrevoir. Ni le toit ni
l'extrêmité nord du temple n'était visible malgré le matin qui avait
maintenant étiré ses bras. L'extrême richesse des décorations,
tableaux et sculptures révélées, mais aussi les vaisselles,
instruments de musique et livres en tout genre, montraient que le lieu
devait être fréquenté, peut-être même habité. Mais les dorures, les
tapisseries et les bibliothèques ne s'étendaient que sur une centaine
de mètres, et derrière, les murs devenaient nus. Ce vide semblait
alors s'étendre sur une distance infinie, car l'aurore ne semblait pas
pouvoir chasser la nuit qui attendait au fond de la pièce.

En voulant refaire ses lacets, Ayno jugea préférable d'enlever ses
bottes baignantes que la semelle droite trouée rendait
désagréables. En regardant la porte d'entrée, il vit qu'une petite
cascade se formait sur les marches de l'entrée du temple, et que des
petits bras d'eau embrassaient les colonnes qui l'avaient abrité aux
premiers temps de la tempête. Dehors, la place était encore déserte,
que le soleil effleurait tout juste. Il referma vite cette image en
ramenant avec grand effort vers lui les deux battants de la porte
d'entrée pour empêcher que cette cascade ne se répandît et finît
par attirer l'attention sur cet endroit.

En verrouillant cette nouvelle écluse, Anoy ramenait à lui une masse
d'eau. En même temps, une vague d'obscurité à laquelle ses yeux
étaient plus accoutumées vint s'abattre sur l'intérieur. En se
retournant pour voir cette vague qui ne revint jamais, l'image du ciel
descendu à ses pieds lui apparut. Il vit, comme autant d'étoiles
terrestres, les centaines de flèches que pouvait contenir son carquois
échappées, déversées partout dans le temple. Trempées dans le fier
acier des forgerons de sa famille, elles possèdaient une qualité
luminescente unique, qui leur permettait de refléter le moindre éclat
au centuple. Le peu de lumière qui traversait les quelques vitraux
situés à plusieurs dizaines de mètres de Anoy, reluisait en centaines
d'étoiles à ses pieds. En voulant ramasser quelques unes d'entre
elles, il comprit qu'elles étaient sa seule source lumineuse à
l'intérieur. La lumière du ciel ne semblait pas vouloir se diriger ici
bas.

Il crut entendre un bruit. L'avait-on vu ?  Anoy se fige, dans un
moment de détresse intense, il crut entendre son père. Une voix
intérieure se mit à parler. En tendant par réflexe son oreille, il
crut entendre qu'on le regrettait au pays; l'Inconnu de son esprit se
réveilla, et renferma la porte. 

Et finalement, une voix finit par ressortir, mais depuis quelque part
dans le temple. Il vit qu'elle provenait d'une lyre trônant en haut
d'un promontoire que le vent avait mystérieusement épargné. En
approchant l'oreille de cette dernière, il vit que cette dernière
était capable de parler avec ses cordes. "Tu es la première source de
chaleur que je rencontre depuis que j'ai quitté l'âtre de mon
foyer. Les quelques auberges dans lesquelles je me suis encanaillé,
tous les paysages si sublimes étaient-ils que j'ai pu traversés, n'ont
pas réussi à apporter une once de chaleur à mon coeur que tu as comblé
en un instant."  Il sent que les émotions d'icelle n'avaient pas une
texture humaine ; il lui semble qu'elle fuit. Il prit la lyre géante
et la plaça à la place de son arc qui avait disparu.

Revenant à lui, Anoy entend cette fois-ci distinctement un bruit
provenant de la porte. Elle lui paraît trembler. Le niveau d'eau, qui
ne cesse de monter, semble faire maintenant un contrepoids trop lourd
pour permettre l'ouverture.

Anoy détourne le visage de la ligne jaunie qui arrive à se faufiler
dans l'entrefilet de la porte d'entrée. Il doit fuir avant qu'on ne le
découvre au milieu de cette profanation. La seule issue se situe
maintenant de l'autre côté du temple ; la montée des eaux provient
nécessairement de cet endroit.


