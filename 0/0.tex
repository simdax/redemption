Dans le salon dans lequel il reçoit ses invités, par pudeur, le coeur
préfère montrer à ses hôtes ce qui fait briller plus l'oeil que la pensée.

Avant de leur servir les légumes les plus saisonniers, les entremets
les plus parfumés et les vins les plus enivrants, un hôte digne de ce
nom lit une sélection de poèmes rimant par paires, tierces et toutes
sortes d'homothéleutes, au rythme de musiques vibrant dans des
rapports inférieurs à cinq. Le repas peut commencer, et ne
s'interrompra pas avant que le coeur n'ait exposé les plus profondes
observations qu'il n'ait faite dans la journée, selon l'intérêt de
chacun, et tiré l'intégralité des principes qui en découlent, des plus
naturels aux plus complexes selon la familiarité que chacun possède
avec les différentes logiques réelles et imaginaires. Quand l'heure
est trop tardive, s'ils sont repus ou simplement trop à l'aise, il
invite ses hôtes à s'endormir; par politesse, et par respect pour leur
fatigue de la vie, il s'excuse de les faire s'allonger dans les plumes
les plus soyeuses, et choisit de les coucher dans des draps blancs ou
safranés. Il les recouvrent pour tout tissage des parures lointaines
dont le sens véritable, incantation païenne, hécatombes, guerres
meurtrières ou danse de fertilité frénétique, prévaut moins que leur
origine, couverture lointaine, que leurs couleurs, leurs silhouettes
dans les ombres que le feu dessine.

Les motifs, les icônes et autres runes antiques qui les recouvrent
font référence aux temps légendaires. Les causes se rapportent aux
effets les plus mythiques. Les raisons les plus intelligibles et les
sentiments les plus nobles y sont peints dans des couleurs inouies
ceints de murs invisbles. Le palais comportant de telles chambres est
bâti sur la terre selon le plan le plus géométrique pour que la vérité
qui pourrait potentiellement en sortir soit faite de l'authenticité la
plus pure.

## io io

Il est une cité dont l'aura religieuse sur son temps fut infinie. Elle
est située dans une plaine riche, fertile en toutes céréales. Ses sols
sont sucrés par tous les fruits, entre un fleuve plus large qu'une
forêt et une forêt plus profonde qu'une mer. La cité prospère depuis
qu'il existe des mémoires pour faire des récits, et pourtant, personne
ne l'a jamais dirigé. Aucun roi n'a pu s'inscrire dans ses registres,
et actuellement personne n'y fait autorité. Comment ont pu s'élever
dans cet endroit les murailles les plus élégantes, les tours les plus
belles, si personne n'y a appliqué une volonté, si aucun fondateur n'y
a enfanté son peuple?

Cette place existe, et il ne faut pas croire qu'un dessin trop parfait
ou trop caricaturale n'est autre que la maladresse d'un conteur à
l'imagination trop pauvre. La seule indulgence qu'il pourrait
revendiquer est qu'il est difficile de tirer la moindre information de
ce pays, car ceux qui y sont allés n'en parlent jamais.  Personne qui
n'y a fait voyage ne peut deviner en quoi consiste la religion qui est
le ciment démocratique de la cité.

Il ne faut pas croire non plus que des dessins moins lumineux n'ont
pas leur place. Il existe aussi

Pour parvenir jusque-là, il faut nécessairement faire un long voyage,
quelqu'en soit le point de départ. Anoy a fait ce long voyage, dont
les histoires ne sont pas belles à raconter, et que son coeur
évite. On dit que les voyages forment la jeunesse, mais pas celle-ci.

Son voyage lui a appris à se méfier. Son coeur est maintenant habité
par un Inconnu qui y a élu domicile. Il règne d'une main de fer sur
son âme, et lui impose le silence. Pourtant, Anoy est dorénavant sur le
bon chemin. Tout le monde lui indique la même direction, et il est peu
probable que tous se trompent.

