Voyez-vous, Samarcande, mon "problème", enfin, si l'on peut appeler
cela un "problème", non, attendez, mon expression est mal choisi. Un
problème, c'est plutôt de ma part une assertion, peut-être même une
apodictique, mais pas un problème, je bute à présent, je souhaiterais
signifier par un mot convenu un thème, une idée, quelque chose que je
jouirais vous voir contempler tel un paysage, et dans ce paysage un
"point" que je voudrais aborder, donc, oui, en tout logique je devrais
plutôt parler de mon "point", enfin quoi que ce terme connote la
métrique terrestre, et je ne vois pas ce qu'il pourrait y avoir de
commun entre un point, une ligne ou une surface humaine, purement
humaine, et un carré

--- hmm.

--- ou un cercle.

--- ah

--- ou un polyèdre ma foi quelconque,

--- vous aussi monseigneur, vous cédez aux sirènes de cette nouvelle
mode.

--- quelle mode?

--- toutes ces nouvelles manières de compter qui se font en marquant
les nombres dans la terre.

--- Ah cela. Tiens, oui, oui, mon bon Samarcande, que voulez-vous,
vivons avec notre temps. Mais, remarquez, je n'ai pas tellement
cédé. Je continue aussi "à l'ancienne".

--- Oui, 

--- oui

--- Certes.

--- naturellement

--- Donc, quel était votre "point", monseigneur ?

--- mon problème c'est que je manque de confiance en l'humanité.

--- ah bon? Vous?

--- oui, enfin, pas exactement, mais j'ai besoin de clarifier ma pensée.

--- Faites, faites

--- Enfin... Je ne sais plus ce que je voulais dire.

--- Si, si, votre seigneurerie ne pourra que se remémorer une si
excellente pensée comme devait l'être celle qui a sonné à votre porte
il y a quelques centaines de secondes.

--- oui, ça, par exemple, la "seconde". Que cela signifie-t-il, sinon
que quelque chose arrive en "deuxième" ?

--- oui, on peut voir cela sous cet angle, mais, l'on peut dire aussi
simplement que c'est quelque qui "suit", qui "vient après" pour dire
les choses autrement. Une sorte de fidèle serviteur de l'instant pour
parler crûment.

--- je vois qu'en tout instant tu ne perds pas une occasion pour
parler de tes qualités propres, mon bon Samarcande.

--- oh non, vous répondrais-je en toute humilité.

--- Malgré toute l'admiration que je peux avoir pour tant de sagacité,
j'ai cru remarqué que la chose qui comptait vraiment le plus pour la
vision du monde, qui donnait le plus de couleur à notre vision du
monde, c'est la confiance que l'on a dans les personnes avec qui l'on
parle.

--- Ne penseriez pas simplement que cette saute d'humeur vous est dicté
par une prise de conscience de vôtre âge avançant monseigneur?

--- J'ai l'impression qu'une dissociation terrible se créé dans notre
monde, que les moeurs de notre temps ne se contentent plus de la
vérité simple, mais tente de faire apparaître quelque chose de plus
puissant. 

--- vous savez, la croyance d'aujourd'hui sera la morale de demain.

--- c'est de vous, ceci cela, mon bon Samarcande ? Non, je ne pense
pas, vous me ferez parvenir au plus tôt l'adresse de la personne qui
aura sifflé ce mot, je vous en serai très reconnaissant.

