Ce n'est pas tenable d'avoir une envie précise en tête, mais pas de
moyen concret pour l'accomplir. En plus, dans ces cas-là, on finit
toujours par mentir. Lorsque Carri demanda à Anoy ce qu'il comptait
finalement accomplir dans la Cité, ce dernier ne voulut pas le
décevoir. Devait-il mentir pour correspondre plus fortement à l'image
qu'il voulait que Carri et toute cette petite famille improvisée sur
la route se fassent de lui? Qu'était ne pas mentir pour le dernier
enfant d'une race perdu dans le monde, et ne voyant le mouvement des
choses que comme un accélérement?

--- Je souhaite intégrer une 


Arrivé à l'endroit dans lequel ils habitaient, Carri et Aarond
proposèrent à Anoy de monter dormir chez eux. Son projet d'aller se
faire moine pourrait bien attendre la nuit, et il était de toute façon
trop tard pour aller agir à cette heure-là. Mais non. Anoy voulait se
retrouver seul. Il savait où les trouver toujours au cas où. Il les
embrassa, vola une plume de Tchip qui s'était perdue dans le dos
d'Ovaline, la glissa dans ses cheveux. Il ne voulait pas partager
quelque chose qui pourrait ressembler à de l'émotion. S'étant assuré
que décliner l'invitation n'était pas trop impoli, il prit congé.

Ainsi en allait-il avec ses histoires de suicide, qui restaient dans
le vague. La pensée concrète de son sang qui se déversait donnait la
nausée à Anoy, tout en produisant un charme si réel sur son esprit,
qu'il n'arrivait pas à détacher ses yeux de lui-même, de son intérieur
rouge. En fermant ses yeux, il sentit son corps baigner dans une
douleur assez tiède, et il imaginait un espace aquatique qui
l'emplissait, comme une errance dans une ville nouvelle. Cela le fit
culpabiliser, mais l'émotion se noya elle aussi.

Anoy rangea sa plume dans l'intérieur de sa manche droite. Il ne
voulait pas se faire remarquer par le silence. Les quartiers
extérieurs luisaient d'un bleuté étrange, irréel. Les rues étaient
vides, plongée dans une ambiance religieuse, pas de bruit, seules
quelques impressions murmurées toujours au bout. Une rumeur, lente et
verte, s'imbriqua dans les pierres. Un oiseau passa, grand et
menaçant, fit regretter son choix à Anoy d'être arrivé ici.

Le quartier sur lequel il déboucha ne cessa pas de l'étonner. La
rumeur était celle d'un grand bâtiment, de taille bien supérieure à
ceux qui le bordaient. Il en surgissait une musique pas inconnue, un
chant de femmes et d'hommes, vaguement moderne aux oreilles
d'Anoy. Dans un premier réflexe, il alla contre le tempo et rebroussa
chemin. Dans la rue par laquelle il était venu. Derrière lui, l'oiseau
marqua une arrêt et une onde s'imprima musicalement aux cris
lointains. Puis finalement, il se mit à nouveau à rebours. Il ne
savait de toute façon pas où aller.

La maison est en bois tandis qu'autour d'elle un torchis sale orne la
lumière hypnotique d'une ville de crépuscule qui imbibe le regard
étranger. Comme hypnotisé, Anoy pénètre dans la maison dont la porte
se détache nettement par un vitrail de très belle facture. Il
représente un lézard. Un grand tapis, de facture assez ancienne,
symbolisant un couple enlacé dans les vagues, un groupe de petits
chiens au premier plan, accueillait le visiteur. Les vagues se
concrétisaient parfois réellement dans le mouvement du tapis, secoué
par les bruits de l'extérieurs, devenus musique de l'intérieur; de ce
petit sas, on aboutissait à une salle bien plus vaste, elle aussi
richement tapissée, grand tapis jaune et bleu sur lequel dansaient sur
la musique des chambres deux hommes, chacun la tête posée sur l'épaule
de l'autre.

Un homme élégant surgit de l'escalier. De son pantalon, une bosse
démesurée donnait du relief à sa démarche.

Bonjour, cher ami féal. à n'en point douter tu ne t'es pas trompé
d'adresse. Es-tu venu pour un exercice spirituel ou pour un
simple renseignement sur la Foi?

Surpris par l'élégance du camarade, Anoy se surprit à accorder sa
confiance à l'être féérique.

Vous voulez dire, vous êtes capable de me renseigner sur ma Foi?

Je n'ai pas dit, votre Foi, j'ai dit la Foi, il me semble que vous
avez tendance à tout confondre, m'amy. C'est à n'en point douter le
premier problème qu'il nous faudra aborder ensemble.

Sous le charme du personnage, Anoy anonna:

Donc, ici, l'on discute de ce genre de question. Et, tous les sujets
peuvent s'aborder ici, qui ont un certain rapport avec cette question?

Peut-on parler de choses, comme... la mort?

Un grand râle mâle surgit d'une pièce supérieure. Saint-Amant fronça
les sourcils.

On ne fait que ça, parler de ce genre de choses. En revanche, il est
possible que dans l'immédiat, il me faille m'absenter. Je ne sais pas
combien de temps cela va me prendre, vous m'en voyez contrit; faites
comme vous le souhaitez, vous pouvez m'attendre ici, ou simplement
revenir un autre jour. à n'en point douter, vous me trouverez plus
disponible.

Et il disparut comme il apparut, vocalement accompagné.

Anoy posa son séant sur un divan, en méditant l'expérience qu'il
venait de faire. Il contempla la danse des deux hommes, qui s'étaient
mus d'une vingtaine de centimètres en autant de minutes, sans sembler
expirer une seule fois tout en ne cessant jamais d'inspirer à grands
poumons.
