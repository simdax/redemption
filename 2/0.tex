De loin, le soleil vacillait. Sa turbulence ondoyait dans le coeur des
voyageurs tentant de pénétrer la ville. Il disparaitraît bientôt,
laissant chaque voyageur tentant de pénétrer la Cité seul avec son
désir de pavé et de gloire.

Dans un dernier effort il irisait, maintenant rougeoyante la route. La
ligne ocre de l'horizon détachait plus droites les tours de la Cité;
par quelques détours la cible devait maintenant être accessible
facilement. Carro estima qu'ils arriveraient dans une heure, juste au
moment de la tombée de la nuit.

Cette nouvelle excita un seul instant Anoy, si tu me permets, lecteur,
de supposer qu'il est possible de découper le temps avec les
instruments que les bouchers et les soldats usent dans leur besogne.
Dans la porte du coeur, cet espoir fut aussitôt refermée par le grand
Inconnu qui la protégeait. La peur, Anoy figé, il décida de jouer quelques
instants avec Ovaline, qui avait quitté la deuxième fourgonette qui
faisait office de volière.

Anoy regardait depuis le début du voyage la petite jouer avec ses
compagnons ailés. Il sourit en pensant que la qualité de l'air remué
par le battement constant des particules de plume avait donné à cette
petite fille une légèreté de coeur qu'il ne rencontrerait peut-être
jamais plus. Elle volait pour donner une graine à un persignol, elle
bondissait en élastique à la manière de ses oiseaux qui tente de
s'approcher prudemment du danger. Son épaule effleurait une cage,
remuait un hibou de race inconnu, grand et effilé, plongé dans le
sommeil.

décidé de monter dans la charette de Carro pour la derière partie du
voyage. Elle avait gardé près d'elle tous ses oiseaux, même Tchip, qui
ignorait maintenant le musicien dans le sommeil.

On fait de la muvique? demanda gentiment Ovaline. Mev oiveaux
f'ennuient, v'ai l'impreffion.

Anoy, qui faisait semblant de dormir, ouvrit un oeil.

Ils ne m'ennuient pas.

Il F'ennuy. Eux ! Tu penfes touzour à toi même ! Alez, on fait de la
muvique.

A nouveau terrifié par la demande, Anoy analysa toutes les
possibilités d'y couper court, mais sans succès. La lyre était là, à
portée de mains, un oiseau qui s'était réveillé alla frotter ses
plumes contre les cordes dont la résonnance fut sublime, et même Carro
dont l'oreille traînait y alla de son petit : ``oh oui, tiens. ça
serait chouette.''

Je ne suis pas un musicien, pensa intérieurement Anoy. D'ailleurs, cet
instrument n'est même pas vraiment à moi. Sans savoir pourquoi, la
chanson qui l'avait réveillé de son cauchemar lui revint à
l'esprit. Il se dit qu'il pouvait bien arriver à contenter tout le
monde avec elle. Au fur et à mesure qu'il y pensait, il imaginait
comment pouvait se dérouler la chose. Il pouvait rater misérablement,
il pouvait réussir ce qu'il avait à faire. à force d'y penser, l'envie
de faire la chanson s'installa définitivement. Les petits yeux
d'Ovaline et même les petits oiseaux qui lui volaient dans les plumes,
lui donnèrent envie finalement de s'y confronter.

Il saisit la lyre, et, comme par enchantement, chaque corde qu'il tirait au
hasard s'y accordait, parfois assez élégamment, souvent assez
platement, mais sans aucune fausse note trop grave.

--- C'est un mode musical du pays dont je viens, si vos oreilles
n'étaient pas encore habituées...

Mais il n'y avait rien à faire, Anoy avait trop envie de le faire.

*chanson*


--- Continue de chanter pendant que l'on passe la muraille, avec un peu
de chance, ça nous évitera les questions un peu embarassantes.

A l'audition de la charette musicale, le douanier roux et sot de la
porte Sud appela son collège blond et pas beaucoup plus malin.

--- regarde, Minie arrive à boire son lait maintenant ! 

Il parlait d'un petit chat abandonné qu'ils avaient trouvé la veille,
et qui se nourissait peu. Le chat miaula à l'écoute de la lyre, la
première fois depuis hier.

--- ooh ! Minie a miaulé !

Le regard inquisiteur se changea en sourire, pendant que chaque
jappement du chat s'inscrivait en contre-temps. Puis, un deuxième
miaulement se fit entendre derrière, puis un troisième et ainsi de
suite. 

Quand la charette fut à l'arrêt, une bande de chats sauvages, dont
Minie, bondirent en face de la charette. Ils commencèrent à
l'attaquer. Anoy cessa immédiatement de jouer, mais il comprit bien
vite que ce n'était pas ses talents de musicien assez médiocre qui les
attirait, mais bien les oiseaux l'accompagnant en rythme qui avaient
attiré leur curiosité.

Les chats avaient une habileté
évidemment bien supérieurs à Anoy, Ovaline et Carro, et tous les
oiseaux, voyant l'imminence du danger prirent la fuite. Une bonne
raclée mit les derniers chats en déroute, mais Ovaline était d'une
tristesse infinie.

Pleins de doute et attristés, Anoy et ses compagnons pénétrerent la
grande cité aux murs immenses, mais cette fois-ci ce n'était pas en
rêve. Cette confrontation avec la réalité le gratte. Il sentit que son
envie de suicide, dans le contexte vague du voyage, était inoffensive,
le passage entre les lieux et l'inconnu étant eux-même une sorte
d'au-delà. Mais l'excitation des molécules provoquée par la friction
du réel réveillait en lui quelque chose de plus fort qui le saisissait
comme une envie de miauler.
