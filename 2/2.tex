Au bout d'une demi-heure, ne sachant pas trop s'il devait y aller ou
non, s'il était étrange qu'il restât encore ici ou oui, de ce qu'on lui
dirait si on le trouvait allongé sur le canapé à moitié endormi,
imaginant l'homme qu'il venait de voir le réveiller avec sa bosse, il
tenta de se lever, mais il craignait de gêner le couple en face de
lui qui n'avait toujours pas bougé. 

À chacun des pas du couple moussait, velours, une bulle sonore, qui,
après plusieurs minutes avant qu'un nouveau grain ne s'ajoute à la
grappe acoustique des remous de la maison, éclatait en éclat de
silence caressant, exhalant un air frais de plaisir. Le sable des murs
s'écoulait lentement des murs emportant avec lui et en secret les mots
échangés dans les chambres. Les petits cris qu'Anoy n'entendait pas
pour la première fois, il n'arrivait pas à les prononcer, car c'est
une autre caractéristique de sa race, en plus de ne pouvoir se
suicider, qui est de ne pouvoir prononcer de petits cris. Tandis que
la première est morale, la seconde est simplement biologique. La
phonation est un phénomène complexe, et il requiert aux vivants une
grande quantité de muscles, qui, ne se trouvant pas au bon endroit au
moment précis que la Nature avait prévu, empêche l'émotion de sortir.

Cette dernière parole était donc inconnue d'Anoy, qui piquait de
curiosité à chacune de ses itérations. C'étaient des cris d'hommes et
de femmes, des cris d'arômes et des cris d'âmes. Il n'était pas naïf,
il comprenait.

Puis, dans ces embrassades voilées, des sons secs et boisés, quelques
cahots de l'ombre, contrastes qui narrèrent une grande femme noire aux
cheveux vert. Sa peau était de cuir, ses vêtements étaient de
peau. elle semblait nue dans le mouvement et l'air vague et pressé
qu'elle remuait éméchait son ovale visagier d'un flot de mystère. Mais
on sentait à la courbure des muscles de ses jambes l'expérience de la
marche et de la course. Son regard fixe qu'Anoy peina à apercevoir
entre les murmures criés rappelait une branche d'arbre avant qu'elle
ne brûle. En voulant s'y raccrocher, un des hommes dansant, à son
apparition, releva lentement la tête.

--- Au revoir, Mangalore.

--- Au revoir, les jumeaux

En passant devant Anoy, la démarche de Mangalore se stabilisa. La
texture du corps apparut, et, excepté le visage, aucune partie de peau
n'était en fait visible. Elle rit, le regard toujours fixe et
intourné, sans attendre une réponse à son rire; soudain, Saint-Amant,
surgi par une porte cochère, cria : ``Tiens, tu es toujours là
toi. Dis-moi, prends le jeune homme s'il-te-plaît. À n'en point
douter, il trouvera avec ta compagnie de quoi répondre à ses
questions.''

Une barre apparut sèche sur le visage de Mangalore. 

--- Je ne suis pas tendre, qu'il le sache.

--- Ne sois pas si perfide, s'il-te-plaît.

Les jumeaux cessèrent de danser. Le jumeau qui ne voyait pas la scène
fit un pas de tango qui instilla une dernière ronde qui renversait la
symétrie et symétrisa le renversement. Il ouvrit son champ de vision
de toutes ses forces. Le grincement des planches de bois sous leur pas
n'hésita plus. Après un gros craquement de bois qui brisait la douce
harmonie des lieux, on entendit l'autre jumeau à nouveau aveuglé
chuchoter : ``tu me diras ce qu'il s'est passé.''

--- Bon, mon grand, de quoi veux-tu qu'on parle ?

Tétanisé par la soudaine attention qu'il recevait, Anoy bredouilla
cette phrase:

--- Pouvons-nous sortir ? Je pense qu'il y aura une ambiance plus
propice pour la conversation ailleurs.

Il attendit une réponse, le doublon rondoya une nouvelle fois. La
nouvelle paire d'yeux dit : ``je te dirai tout.''

--- Tu connais ma réponse? demanda-t-elle.

--- Non.

Ne sachant pas s'il avait répondu bon ou non, il suivit:

--- Vous êtes certaine que cela ne vous gêne pas ?

--- Écoute, je suis pressée de rentrer chez moi, je suis ici parce que
je rends service à un supposé ami, mais n'abuse pas.

--- Je ne suis absolument pas capable de mener une discussion si je ne
suis pas certain d'un certain degré d'aisance dans la conversation.

Saint-Amant surgit.

--- Très bien, voilà ce que je voulais entendre. Écoute, Mangalore, je
ne suis pas sans ignorer que notre relation est d'une parfaite
stabilité, qui autorise, pour répondre aux lois de la perfidie, un
certain écart salutaire. Prends cela en mille comme en cent, nous
sommes quittes pour la prochaine fois où tu me devras quelque
chose. Vois-tu ce que je veux dire? Accompagne ce jeune homme dans
l'intimité de sa question.

On aurait pu croire, aux quelques mots prononcés précédemment que
Mangalore fut pris d'une irrésistible envie d'envoyer son ami
Saint-Amant bouler, et pourtant les jumeaux se mirent à danser
frénétiquement. Comme ces derniers étaient le métronome de l'humeur
des lieux, Mangalore, cette dernière, inspira profondément, tourna
pour la première fois de la soirée la tête vers un autre être vivant,
son ami, son frère. Elle prit son bras et l'apposa sur sa poitrine.

--- tu entends mon coeur battre, frère?

--- ne sois pas si perfide, s'il-te-plaît.

--- Tu veux vraiment que je monte avec l'autre ange?

Mangalore fixa Anoy dans les yeux.

-- allez, viens, frère angélique.
