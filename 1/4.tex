--- alors le musicien ? Ah, enfin, content de le voir, vous êtes enfin
réveillés ? Quels paresseux ceux-là ! Et toujours à faire la
concurrence aux oiseaux.

La moustache qui venait de s'exprimer portait sous un chapeau de luxe
des habits de ville, ce qui n'était pas le cas la veille, dans les
vagues souvenirs qu'Anoy pouvait collecter dans sa mémoire. Le
vieillard s'était changé, ce qui signifiait que D. n'était vraiment
plus loin d'ici, et qu'ils arriveraient aujourd'hui-même.

Ils étaient en train de faire une pause. Pour faire tant de chemin, il
avait dû rouler toute la nuit, pendant que lui s'imaginait les pires
choses qui soient. Anoy culpabilisa instantanément.

--- Tu as sorti ton instrument finalement, tu fais tes gammes ? C'est
bien, au moins yen a qui travaille ici !

Anoy s'excusa. Que pouvait-il faire allongé dans l'herbe à regarder
les oiseaux tandis que la personne qui avait accepté de lui rendre un
si grand service, si âgé par dessus le marché, était en train de faire
quelque chose certainement d'important. Comment avait-il pu se
retrouver sous cet arbre d'ailleurs ?

--- Et la petite qui adore aussi la musique, à part chanter avec les
oiseaux, elle ne fait pas grand chose, mais c'est de son âge
va. Tiens, c'est quoi ton nom déjà ?

--- Anoy

--- Oui, Anouille, montre-lui des choses, tu veux ?

De derrière la charette, une petite fille qui avait totalement échappé
à Anoy sortit, tenant au bout de ses doigts trois oiseaux, dont Tchip,
qui s'enfuit instantanément à sa vue.

--- oh... qu'est-fe qui lui arribe ?

--- qu'est-CE qui lui arriVe ! on n'y arrivera jamais avec cette
petite ! Si tu pouvais lui apprendre à articuler aussi, ça serait pô
mèl non plus tiens. ça serait bien la déveine qu'elle nous trimballe
ça toute la vie.

--- Falu, moi c'est Obaline

--- o-VA-line, prononce bien ton prénom au moins !

--- Mais f'est fe que j'ai dit ! oBAline !

Toujours sous le choc de l'apparition du petit être gazouillant, Anoy
tenta vainement de recoller les morceaux dans sa tête. Où avait-il pu
dormir, pour que

--- montre-moi, montre-moi ! Ze peux zouer ?

Anoy se retourna et vit la lyre qui était bel et bien là. Il ne voulut
pas la prendre dans ses mains, de peur de se retrouver à étouffer, ou
alors collé à l'instrument. De plus, il ne se sentait absolument pas
musicien, et il ne voit pas du tout quelle chanson il aurait pu jouer.

Plus vite que pour dire ouf, il vit qu'Ovaline s'était précipité sur
l'instrument sans attendre une quelconque confirmation.

--- ça va ? Tu... tu peux respirer ?

--- bah bien fur. Ovaline exhala un grand souffle qui fit frissoner la lyre.

--- oooh ! F'est trop voli ! 


Une deuxième charrette sortit du fond de la clairière.

-- Ah bah, tu en as mis du temps !

Un deuxième vieillard

Oui, je ne voulais pas amener le petit au marché de Z. vu l'état dans
lequel on l'a retrouvé hier. Il est réveillé d'ailleurs ?



--- Moi, c'est Arond.

--- Et moi, c'est Carro.



