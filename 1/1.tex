--- La plus grande félonie sur Terre se nomme le sexe féminin.

--- Vous faites allusion aux menues petites catins quelconques qui
se trouvèrent hasardeusement sur votre auguste chemin, monseigneur ?

--- S'il-te-plaît, va me chercher du jambon sur cet os encore bien
dodu qui se trouve là-bas. Non, je ne veux pas parler de ces
Michelines, pas du tout ; dis que tu as tort.

--- Monseigneur, j'ai tort.

--- Le problème avec toi, c'est que tes remarques me font toujours
perdre ma concentration. Que disais-je, déjà ? Tu es distrayant, mais
veuille réserver tes commentaires uniquement au moment opportun de la
prochaine fois, s'il t'enjouit.

--- Vous parliez de félonie féminine.

--- C'est cela, oui. Bravo, quand même. Tu caches bien ton
jeu. Hum. Oui, ce que je voulais dire, c'est que ce n'est rien d'autre
que la naturelle perfidie féminine qui durcit le coeur des hommes.

L'assemblée, qui se permettait encore de minuscules échanges, se tut
totalement.

--- Ce coeur si prompte à réfléchir joyeusement sur le sens de la vie
quoi de plus beau, je vous pose la question, que de disserter sur la
laideur en étant soi-même beau ?  Quoi de plus masculin que d'accorder
tant de grâces à si peu de choses ? La beauté qui ne se montre pas,
l'harmonie infinie au milieu d'un chaos pourtant étendu bien au-delà
de ses frontières... Mais dont il dépasse les frontières par la simple
existence de son englobance, c'est cela, l'essence et la véritable
condition humaine. C'est une proposition que je soumets en public pour
la première fois, je comprends votre silence soudain -- excusez-moi,
Samarcande, qu'est-ce que je disais ?

--- Vous parliez de félonie féminine.

--- Ah oui, félicitations. Veuillez me chercher un peu de jambon à
l'os que je vois trôner au bout de cette table exquise, empli de tant
de mets, et ceintes de tant de seigneurs si féaux, et de dames-oiselles si
jolies. Mesdames, ne prenez pas mal ce que je suis en train de dire,
je vois aux regards de certaines que je vous vole dans les plumes.

L'assemblée rit.

--- rou rou, rou

Samarcande approcha du seigneur Landis avec sept tranches de cuisses
découpés en suivant la ligne osseuse.
