--- Cette fois-ci, tu n'auras pas besoin de me rappeler ce que j'étais
en train de dire mon bon Samarcande, car, vois-tu, je suis entré dans
mon sujet maintenant. C'est cela, les lignes de mon discours sont
définis selon la modalité, la qualité et la relation, parfait, j'y
vais. Ce coeur mâle donc, si beau et si doux, si plein de charmes et
de perfections dans sa détresse même, quoi, quelles choses dans la
nation peuvent-elles arriver à nous le corrompre? Car, à n'en pas
douter, l'homme est corrompu. n'est-ce pas, beaux et bons seigneurs
bien féaux, et vous, demoiselles ceignant si élégamment cette table
emplie de mets si exquis. Il n'est qu'à voir sa barbe qui pousse
malgré le bon sens. Qu'est-ce que le bon sens, demanderez-vous,
seigneurs et demoiselles. Voilà ce qu'il est, ce sus-dit bon sens: je
ne veux pas d'une chose, j'arrête d'y penser. C'est aussi simple que
cela. Je ne veux pas manger, je détourne mon visage vers un autre lieu
où me guident mes yeux. N'est-ce pas le bon sens le plus samarcandien?
Je vois que mes hôtes esquissèrent un sourire à l'évocation de votre
nom, je voulais louer votre féodalité. Pourtant, si je désire que ma
barbe cesse de pousser, cela ne se peut-il? Y a-t-il un moyen, ou cela
ne se peut-il pas? Par quels moyens se fait-il donc, que le coeur de
l'homme, malgré ses grandeurs dont personne n'a encore vu le fond, se
puisse corrompre vers l'état cadavérique, tel un végétal qu'on
cesserait d'enluminer? Quelle cause pour une conséquence si funeste,
si ce n'est par la déception totale qu'est la félonie féminine ?

--- Rouuu, kokek

--- La femme qui, je le rappelle, est si proche de l'homme, lui est
son égal à n'en pas douter, lui est similaire sur le plan de l'âme et
sur l'oblique de l'intelligence, et cela, en absolument tous les
points du plan, cela est confirmé par l'expérience la plus commune, la
plus quotidienne. Combien de fois même, n'ai-je pas ressenti, presque
vu, trompé par les sens, en une femme, un homme sans douceur qui s'y
cacherait ? Si ce n'est par la raison ou par les émotions, je le dis
tout net, c'est par la morale que les femmes diffèrent. Car oui, alors
que l'homme est porté naturellement, par ce sens si masculin, à relier
les choses et de trouver les liens profonds qui les unit, de bricoler
les idées pour se faire en quelque sorte représentant de son dieu sur
terre, car au fond, il n'est guidé par rien d'autre que par son
sentiment de loyauté, sentiment hautement moral s'il en est, il
atteint, par le dernier stade de la gentilhommerie, l'Idée principale
qui régit toute chose, à savoir, que c'est la raison qui guide le
coeur, et le coeur qui guide la raison.

une vague d'émotion emplit la table

--- C'est pourquoi, chers amis féaux, je vous propose, de briser le
dernier tabou qui empêche la cité de voler de ses propres ailes, je
vous propose et je pense que maintenant votre Raison vous l'exige
aussi, que nous ouvrions l'enseignement religieux aussi aux femmes.

les applaudissements s'interrompirent

--- Nous, qui n'avons pas d'autres ressources que notre force
spirituelle, sommes nous en capacité d'être dans l'avant-garde de ce
qu'est la Foi? Nous, qui avons les écoles les plus réputées dans la
matière, nous, qui avons le modèle le plus équilibré en matière de
tolérance religieuse, nous pouvons maintenant nous permettre de tenter
l'expérience

--- mais il est devenu complètement fou

--- notre équilibre tient justement dans cette injustice ! 

--- Ne crains-tu pas justement que l'équilibre de notre bel cité ne
vienne à s'écrouler à cause de cela?

--- pour quelles


Qu'est-ce que
