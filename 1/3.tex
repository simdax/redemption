La voilet'-lette est comme une marionnet'-nette
Prise par le vent, qui la jette-jette
à terre car il est méchant

La voilet'-lette est en fête-fête
Elle passe par la fenêtre
et atterit sur la maman

Elle l'attrape, caresse sa peau
tout en haut- tout en haut
et la plie en quatre

Dans la poche de la marâtre
tout en haut- tout en haut
Toujours sera le plus beau

Les herbes remuent au rythme de la chanson et l'harmonise avec ses
dissonances et ses secrets .

N'aimes-tu pas cela ?

Anoy ne sait pas où il est. Il sait, il doit être au paradis.

Suis-je mort ?

Réponds-moi

En ouvrant l'oeil droite, Anoy remarqua les jeunes blés. Les fines
tiges qui deviendraient de grands et droits faisceaux adulés par le
soleil. Pour le moment, les jeunes poussent tombent et se redressent
toutes ensemble sous l'oeil admiratif des blés adultes. Le soleil
nourrit abondamment ces dernières d'un éclat lourd.

Ce spectacle céréalier offert ne profite pas à Anoy ; le tournis plein
la tête, il n'a pas la force de sortir de l'ombre où il est semblable
à une plante qu'on protège. Tête lourde contre le sol, il respire les
céréales sauvages et vertes qui irrite sa gorge prise par un rhume des
foins qui se contracte à la lumière mais qu'une énième nuit à la
belle étoile devrait remettre à sa place.

N'aimes-tu pas cela ?

Le torticolis et un mal de ventre nouveau le bloquent au sol. N'ayant
pas même la force de bouger ses mains, Anoy utilise la terre humide
pour réveiller son corps encore endormi. Il caresse ses joues avec
l'humus, et ses genoux avec des racines. Les différents trous de son
linge se mettent à gratter. L'irritation finit par le convaincre de se
retourner, péniblement. Il distingue l'arbre qu'il le tient à l'abri
de la lumière, et derrière lui, une charrette, apparement vide, sur la
toile de laquelle le tronc semble se reposer, lui-même pour gratter les
parties qui le démangent. Sur les roues, un oiseau se pose contre
lesquelles il tape le bec.

Au moment où Anoy décide enfin de se relever, une brise chaude envoyée
par la forêt amène avec elle l'oiseau, qui se pose à une dizaine de
mètres du corps larvé. Peut-être prend-il Anoy pour un énorme ver de
terre ?  L'observation du bec, qui a saisi une paille et deux brins
d'herbe, montre qu'il jalouse l'endroit pour le nidifier. Tout en les
gardant habilement en bouche, il tchippe l'intrus. Il avait reperé
l'endroit peut-être déjà avant. Fixe pendant une petite minute, l'air
contrit, il tchippe une deuxième fois. Anoy reste immobile, pensant
pouvoir s'enterrer, et l'oiseau se jette d'un coup toute aile ouverte
vers Anoy avant de faire brusquement demi-tour. Un dernier tchip
s'éteint sous le bruit d'une feuille qui tombe.

Sans oser le penser, Anoy comprend qu'il n'est pas mort. L'odeur de la
chaleur se pose sur la terre, et malgré lui, lui apporte
réconfort. D'autres oiseaux chantent en fusée au-dessus de lui ; Anoy
tend l'oreille, mais ne remarque aucune tentative de retour du premier
tchip. Avec grand effort, il poursuit ce que la visite ailée avait
stoppée, regagner la terre ferme et quitter le sol.

N'aimes-tu pas cela ?

La charette lui inspire confiance, il sait de quoi il s'agit. Il la
reconnaît. Tout l'épisode de la cité n'était qu'un rêve anticipant
l'arrivée très prochaine. Le voyage était en réalité plus long que
pénible, mais par chance Anoy avait rencontré la veille ce paysan qui
allait en direction de D. et n'avait pas fait de chichis. Transi de
froid et de fatigue, le sommeil qui l'avait gagné cette nuit avait
produit un cauchemar dont il espérait que toutes les bizarreries,
comme les erreurs lors d'une générale, ne se retrouverait pas lors de
la première.

Pris par la soif, en observant de plus près la lisière de la
forêt qui jouxtait la clairière dans laquelle il se trouvait, Anoy
aperçut un fin filet d'eau dans lequel il trouva de quoi nettoyer son
visage et se désaltérer. Son mal de tête, comme une blessure mise à
nue par la rasade d'eau qu'il venait d'ingurgiter, se raviva et Anoy
trébucha presque au contact de l'eau sur sa nuque.

En retournant vers la charette, Anoy lève la tête, et perçoit sur une
branche haute de l'arbre un début de nid ; le nid du tchipeur
possédait en fait déjà une vingtaine de brindilles compactes. Il avait
vraiment  L'oiseau reviendra-t-il terminer ce qu'il avait commencé ?
Peut-être ne reviendra-t-il plus ?  Peut-être la créature a-t-elle eu
trop peur de lui et cherche-t-elle maintenant à s'installer ailleurs ?
Tout cet effort déployé par cette frêle créature n'avait-il servi à
rien ?

Une voix surgit derrière son dos : ``c'est mieux que la mort, n'est-ce
pas ?'' En se retournant, Anoy vit la déesse oblongue et masquée. à
ses pieds, offerte, la lyre se tenait droite au milieu des herbes
maintenant reposées. Tous les frottements des feuilles, tous les
crissements des pierres, la respiration des insectes, étaient absorbés
et ressortaient musique. Anoy lui-même, en approchant, faisait, à
chacun de ses pas, surgir un accord et taire un peu plus la Nature qui
les entourait.

Arrivé au quasi silence, le souffle d'Anoy qui produisait les derniers
sons s'arrêta en saisissant la lyre qui fut le seul instrument qui eut
le droit de respirer. Anoy étouffa. Pris de panique, il craignit
l'instrument qui restait accroché à lui. Il secoua fortement les
mains, et au moment où il voulut le briser contre une pierre, la
déesse le saisit. Ses mains prirent sa main gauche et la positionna de
manière correcte. Anoy qui étouffait toujours, tremblait de plus en
plus. Puis, les mains de la Déesse prirent sa main droite, et
délicatement, l'amena jusqu'à une première corde.

La vibration de la première corde emplit, au sens propre, Anoy d'un
souffle magique. Chaque vibration de la corde rentrait directement
dans ses poumons. Quand la première eut presque terminé de vibrer, la
main de la déesse lui fit tirer la seconde corde. Un nouvelle note, un
nouveau timbre, constitua une nouvelle respiration, elle aussi
inédite. Avant qu'elle ne cesse de vibrer, Anoy tira la troisième. Une
profonde sensation de bonheur et de plénitude emplit ce dernier. La
déesse lâcha les mains d'Anoy et se retira lentement.

Sans bouger, elle posa une dernière question : 

``tu aimes ça ?''

et disparut.

